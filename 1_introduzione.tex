\pagenumbering{roman}
\chapter*{Introduzione}
\rhead[\fancyplain{}{\bfseries
INTRODUZIONE}]{\fancyplain{}{\bfseries\thepage}}
\lhead[\fancyplain{}{\bfseries\thepage}]{\fancyplain{}{\bfseries
INTRODUZIONE}}
\addcontentsline{toc}{chapter}{Introduzione}

Con il continuo sviluppo del mondo inerente agli smartphone si è arrivati alla nascita di nuove tecnologie utili per la comunicazione tra di essi ( Device-to-Device ).
In passato, e anche ora, veniva impiegata la tecnologia del Bluetooth per far comunicare 2 dispositivi smartphone e permettergli quindi di scambiarsi file di vario genere ( musica, immagini, video, ecc… ).
Ma da poco tempo è stata introdotta una nuova tecnologia utile allo stesso scopo che si basa sul protocollo del Wi-Fi: il Wi-Fi Direct.
In questa tesi quindi ci si è occupati di studiare più a fondo lo standard del WiFi-Direct, illustrandone il funzionamento, l’architettura, e gli scenari di utilizzo.
Successivamente sono stati effettuati dei test per determinare la reale efficienza di tale standard, costruendo appositi applicativi e calcolando determinate metriche: Packet Delivery Ratio, Throughput e Round Trip Time.
Per l’esecuzione di tali esperimenti è stato utilizzato il sistema operativo Android.
Nel primo capitolo ci si è occupati di approfondire la tecnologia Wi-Fi Direct.
Nel secondo capitolo di come tale tecnologia funzioni su Android.
Nel terzo capitolo si illustra il funzionamento dell’applicativo creato, degli esperimenti fatti e di come sono state calcolate le metriche.
Nel quarto capitolo infine sono stati inseriti i risultati degli esperimenti fatti.
